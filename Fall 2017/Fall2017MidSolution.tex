\documentclass[]{article}
\usepackage{lmodern}
\usepackage{amssymb,amsmath}
\usepackage{ifxetex,ifluatex}
\usepackage{fixltx2e} % provides \textsubscript
\ifnum 0\ifxetex 1\fi\ifluatex 1\fi=0 % if pdftex
  \usepackage[T1]{fontenc}
  \usepackage[utf8]{inputenc}
\else % if luatex or xelatex
  \ifxetex
    \usepackage{mathspec}
  \else
    \usepackage{fontspec}
  \fi
  \defaultfontfeatures{Ligatures=TeX,Scale=MatchLowercase}
\fi
% use upquote if available, for straight quotes in verbatim environments
\IfFileExists{upquote.sty}{\usepackage{upquote}}{}
% use microtype if available
\IfFileExists{microtype.sty}{%
\usepackage{microtype}
\UseMicrotypeSet[protrusion]{basicmath} % disable protrusion for tt fonts
}{}
\usepackage[margin=1in]{geometry}
\usepackage{hyperref}
\hypersetup{unicode=true,
            pdftitle={Fall 2017 Mid Term Solution},
            pdfauthor={Chen Wang},
            pdfborder={0 0 0},
            breaklinks=true}
\urlstyle{same}  % don't use monospace font for urls
\usepackage{graphicx,grffile}
\makeatletter
\def\maxwidth{\ifdim\Gin@nat@width>\linewidth\linewidth\else\Gin@nat@width\fi}
\def\maxheight{\ifdim\Gin@nat@height>\textheight\textheight\else\Gin@nat@height\fi}
\makeatother
% Scale images if necessary, so that they will not overflow the page
% margins by default, and it is still possible to overwrite the defaults
% using explicit options in \includegraphics[width, height, ...]{}
\setkeys{Gin}{width=\maxwidth,height=\maxheight,keepaspectratio}
\IfFileExists{parskip.sty}{%
\usepackage{parskip}
}{% else
\setlength{\parindent}{0pt}
\setlength{\parskip}{6pt plus 2pt minus 1pt}
}
\setlength{\emergencystretch}{3em}  % prevent overfull lines
\providecommand{\tightlist}{%
  \setlength{\itemsep}{0pt}\setlength{\parskip}{0pt}}
\setcounter{secnumdepth}{5}
% Redefines (sub)paragraphs to behave more like sections
\ifx\paragraph\undefined\else
\let\oldparagraph\paragraph
\renewcommand{\paragraph}[1]{\oldparagraph{#1}\mbox{}}
\fi
\ifx\subparagraph\undefined\else
\let\oldsubparagraph\subparagraph
\renewcommand{\subparagraph}[1]{\oldsubparagraph{#1}\mbox{}}
\fi

%%% Use protect on footnotes to avoid problems with footnotes in titles
\let\rmarkdownfootnote\footnote%
\def\footnote{\protect\rmarkdownfootnote}

%%% Change title format to be more compact
\usepackage{titling}

% Create subtitle command for use in maketitle
\providecommand{\subtitle}[1]{
  \posttitle{
    \begin{center}\large#1\end{center}
    }
}

\setlength{\droptitle}{-2em}

  \title{Fall 2017 Mid Term Solution}
    \pretitle{\vspace{\droptitle}\centering\huge}
  \posttitle{\par}
  \subtitle{Solution for CS 143A course at University of California, Irvine}
  \author{Chen Wang\footnote{Undergraduate in Computer Engineering, Samueli School
  of Engineering, University of California, Irvine.
  (\href{mailto:chenw23@uci.edu}{\nolinkurl{chenw23@uci.edu}})}}
    \preauthor{\centering\large\emph}
  \postauthor{\par}
      \predate{\centering\large\emph}
  \postdate{\par}
    \date{11/8/2019}


\begin{document}
\maketitle

{
\setcounter{tocdepth}{3}
\tableofcontents
}
\hypertarget{basic-page-tables.}{%
\section{Basic page tables.}\label{basic-page-tables.}}

\hypertarget{address-mapping-explanation}{%
\subsection{Address Mapping
Explanation}\label{address-mapping-explanation}}

(10 points) Illustrate the page table used by xv6 to map the kernel into
the virtual address space of each process (draw a page table diagram and
explain the page table entries). Specifically concentrate on one entry:
the entry responsible for the translation of the firstpage of the
kernel. Keep in mind that xv6 maps the kernel into the virtual address
range starting above the second gigabyte of virtual memory. Note, that
after xv6 is done booting,it xv6 uses normal 4KB, 32bit, 2-level page
tables. You also have to recall the physicall address of the first
kernel page (look at the boot lecture or the kernel map), and the
virtual address where this page is mapped. To make the example
realistic, don't forget that xv6 allocates memory for it's page table
directory and page tables from the kernel memory allocator.

\textbf{\emph{Reference Solution:}}

The mapping relationship can be just found on the
\href{https://www.ics.uci.edu/~aburtsev/143A/lectures/lecture10-kernel-page-table/lecture10-kernel-page-table.pdf}{Lecture
10 - Kernel Page Table}, the diagram is on the slide page 79. And for
the page table diagram, this can be found at the slide page 82. As for
the first page of the kernel, it is certainly at virtual address
0x80000000. We take its first 10 bits, and find out that this is at the
512th entry at the page table directory. Then we take the second 10 bits
of the virtual address, which is 0, we can look into the 0th entry of
the page table found throught the 512th entry of the directory. Through
the 0th entry of the table, we find out the physical address of the
kernel.


\end{document}
