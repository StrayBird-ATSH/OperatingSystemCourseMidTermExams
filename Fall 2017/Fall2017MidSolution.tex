\documentclass[]{article}
\usepackage{lmodern}
\usepackage{amssymb,amsmath}
\usepackage{ifxetex,ifluatex}
\usepackage{fixltx2e} % provides \textsubscript
\ifnum 0\ifxetex 1\fi\ifluatex 1\fi=0 % if pdftex
  \usepackage[T1]{fontenc}
  \usepackage[utf8]{inputenc}
\else % if luatex or xelatex
  \ifxetex
    \usepackage{mathspec}
  \else
    \usepackage{fontspec}
  \fi
  \defaultfontfeatures{Ligatures=TeX,Scale=MatchLowercase}
\fi
% use upquote if available, for straight quotes in verbatim environments
\IfFileExists{upquote.sty}{\usepackage{upquote}}{}
% use microtype if available
\IfFileExists{microtype.sty}{%
\usepackage{microtype}
\UseMicrotypeSet[protrusion]{basicmath} % disable protrusion for tt fonts
}{}
\usepackage[margin=1in]{geometry}
\usepackage{hyperref}
\hypersetup{unicode=true,
            pdftitle={Fall 2017 Mid Term Solution},
            pdfauthor={Chen Wang},
            pdfborder={0 0 0},
            breaklinks=true}
\urlstyle{same}  % don't use monospace font for urls
\usepackage{graphicx,grffile}
\makeatletter
\def\maxwidth{\ifdim\Gin@nat@width>\linewidth\linewidth\else\Gin@nat@width\fi}
\def\maxheight{\ifdim\Gin@nat@height>\textheight\textheight\else\Gin@nat@height\fi}
\makeatother
% Scale images if necessary, so that they will not overflow the page
% margins by default, and it is still possible to overwrite the defaults
% using explicit options in \includegraphics[width, height, ...]{}
\setkeys{Gin}{width=\maxwidth,height=\maxheight,keepaspectratio}
\IfFileExists{parskip.sty}{%
\usepackage{parskip}
}{% else
\setlength{\parindent}{0pt}
\setlength{\parskip}{6pt plus 2pt minus 1pt}
}
\setlength{\emergencystretch}{3em}  % prevent overfull lines
\providecommand{\tightlist}{%
  \setlength{\itemsep}{0pt}\setlength{\parskip}{0pt}}
\setcounter{secnumdepth}{5}
% Redefines (sub)paragraphs to behave more like sections
\ifx\paragraph\undefined\else
\let\oldparagraph\paragraph
\renewcommand{\paragraph}[1]{\oldparagraph{#1}\mbox{}}
\fi
\ifx\subparagraph\undefined\else
\let\oldsubparagraph\subparagraph
\renewcommand{\subparagraph}[1]{\oldsubparagraph{#1}\mbox{}}
\fi

%%% Use protect on footnotes to avoid problems with footnotes in titles
\let\rmarkdownfootnote\footnote%
\def\footnote{\protect\rmarkdownfootnote}

%%% Change title format to be more compact
\usepackage{titling}

% Create subtitle command for use in maketitle
\providecommand{\subtitle}[1]{
  \posttitle{
    \begin{center}\large#1\end{center}
    }
}

\setlength{\droptitle}{-2em}

  \title{Fall 2017 Mid Term Solution}
    \pretitle{\vspace{\droptitle}\centering\huge}
  \posttitle{\par}
  \subtitle{Solution for CS 143A course at University of California, Irvine}
  \author{Chen Wang\footnote{Undergraduate in Computer Engineering, Samueli School
  of Engineering, University of California, Irvine.
  (\href{mailto:chenw23@uci.edu}{\nolinkurl{chenw23@uci.edu}})}}
    \preauthor{\centering\large\emph}
  \postauthor{\par}
      \predate{\centering\large\emph}
  \postdate{\par}
    \date{11/8/2019}


\begin{document}
\maketitle

{
\setcounter{tocdepth}{3}
\tableofcontents
}
\hypertarget{basic-page-tables.}{%
\section{Basic page tables.}\label{basic-page-tables.}}

\hypertarget{address-mapping-explanation}{%
\subsection{Address Mapping
Explanation}\label{address-mapping-explanation}}

(10 points) Illustrate the page table used by xv6 to map the kernel into
the virtual address space of each process (draw a page table diagram and
explain the page table entries). Specifically concentrate on one entry:
the entry responsible for the translation of the first page of the
kernel. Keep in mind that xv6 maps the kernel into the virtual address
range starting above the second gigabyte of virtual memory. Note, that
after xv6 is done booting,it xv6 uses normal 4KB, 32bit, 2-level page
tables. You also have to recall the physical address of the first kernel
page (look at the boot lecture or the kernel map), and the virtual
address where this page is mapped. To make the example realistic, don't
forget that xv6 allocates memory for it's page table directory and page
tables from the kernel memory allocator.

\textbf{\emph{Reference Solution:}}

The mapping relationship can be just found on the
\href{https://www.ics.uci.edu/~aburtsev/143A/lectures/lecture10-kernel-page-table/lecture10-kernel-page-table.pdf}{Lecture
10 - Kernel Page Table}, the diagram is on the slide page 79. And for
the page table diagram, this can be found at the slide page 82. As for
the first page of the kernel, it is certainly at virtual address
0x80000000. We take its first 10 bits, and find out that this is at the
512th entry at the page table directory. Then we take the second 10 bits
of the virtual address, which is 0, we can look into the 0th entry of
the page table found through the 512th entry of the directory. Through
the 0th entry of the table, we find out the physical address of the
kernel.

\hypertarget{shell}{%
\section{Shell}\label{shell}}

Alice works on implementing a new shell for xv6. She implements a pipe
command (e.g., ls\textbar{}wc) like this:

\begin{verbatim}
void
runcmd(struct cmd *cmd)
{
  ...
  switch(cmd->type){
  default:
    fprintf(stderr, "unknown runcmd\n");
    exit(-1);
    
    case ’|’: pcmd = (struct pipecmd*)cmd;
      int p[2];
      pipe(p);
      int pid = fork();
      if(pid == 0){//child process:left side
        close(1);
        dup(p[1]);
        close(p[1]);
        close(p[0]);
        runcmd(pcmd->left);
      }
      close(0);
      dup(p[0]);
      close(p[0]);
      close(p[1]);
      wait(NULL);
      runcmd(pcmd->right);
      break;
    }
    ...
}
\end{verbatim}

\hypertarget{wrong-implementation-analysis}{%
\subsection{Wrong Implementation
Analysis}\label{wrong-implementation-analysis}}

\begin{enumerate}
\def\labelenumi{(\alph{enumi})}
\tightlist
\item
  (5 points) Her implementation always waits for left side to finish,
  but she is not sure if it's correct since she notices that the shell
  that xv6 implements (sh.c in the xv6 source tree) launches the right
  side right away. Can you come up with an example for which Alice's
  shell fails, while the xv6's is still correct? Explain your answer.
\end{enumerate}

\hypertarget{os-isolation-and-protection}{%
\section{OS isolation and
protection}\label{os-isolation-and-protection}}

\hypertarget{memory-layout-of-xv6}{%
\subsection{Memory Layout of xv6}\label{memory-layout-of-xv6}}

(5 points) Explain the organization and memory layout of the xv6
process. Draw a diagram. Explain which protection bits are set by the
kernel and explain why kernel does it.

\textbf{\emph{Reference Solution:}}

It can be found from
\href{https://www.ics.uci.edu/~aburtsev/143A/lectures/lecture04-linking-and-loading/lecture04-linking-and-loading.pdf}{Lecture
04 - Linking and Loading}. It is in the page 6 of the slides.

The third bit from the right end is the kernel bit. It prevents the user
from modifying kernel data and prevents potential faults.

\hypertarget{memory-isolation-between-processes}{%
\subsection{Memory isolation between
processes}\label{memory-isolation-between-processes}}

In xv6 individual processes are isolated, specifically they cannot
access each others memory. Explain how this is implemented.

\textbf{\emph{Reference Solution:}}

Each process has its own page table. When reading from a virtual
address, MMU will find physical address by reading this process's page
table. The pages of other processes are not mapped to this process's
page table, therefore, the physical address of other processes will
never be the translate result of this process.

\hypertarget{os-organization}{%
\section{OS organization}\label{os-organization}}

\hypertarget{kernel-base-understanding}{%
\subsection{Kernel Base Understanding}\label{kernel-base-understanding}}

(10 points) \texttt{KERNBASE} limits the amount of memory a single
process can use, which might be irritating on a machine with a full 4 GB
of RAM. Would raising \texttt{KERNBASE} allow a process to use more
memory (explain your answer)?

\textbf{\emph{Reference Solution:}}

Yes. Currently a process can access 2GB of virtual memory because the
upper 4 GB are reserved by kernel. If this restriction is lifted, the
user process can access more memory. Nevertheless, we should acknowledge
that the corresponding page table flags should be changed accordingly as
well.


\end{document}
