\documentclass[]{article}
\usepackage{lmodern}
\usepackage{amssymb,amsmath}
\usepackage{ifxetex,ifluatex}
\usepackage{fixltx2e} % provides \textsubscript
\ifnum 0\ifxetex 1\fi\ifluatex 1\fi=0 % if pdftex
  \usepackage[T1]{fontenc}
  \usepackage[utf8]{inputenc}
\else % if luatex or xelatex
  \ifxetex
    \usepackage{mathspec}
  \else
    \usepackage{fontspec}
  \fi
  \defaultfontfeatures{Ligatures=TeX,Scale=MatchLowercase}
\fi
% use upquote if available, for straight quotes in verbatim environments
\IfFileExists{upquote.sty}{\usepackage{upquote}}{}
% use microtype if available
\IfFileExists{microtype.sty}{%
\usepackage{microtype}
\UseMicrotypeSet[protrusion]{basicmath} % disable protrusion for tt fonts
}{}
\usepackage[margin=1in]{geometry}
\usepackage{hyperref}
\hypersetup{unicode=true,
            pdftitle={Fall 2018 Mid Term Solution},
            pdfauthor={Chen Wang},
            pdfborder={0 0 0},
            breaklinks=true}
\urlstyle{same}  % don't use monospace font for urls
\usepackage{graphicx,grffile}
\makeatletter
\def\maxwidth{\ifdim\Gin@nat@width>\linewidth\linewidth\else\Gin@nat@width\fi}
\def\maxheight{\ifdim\Gin@nat@height>\textheight\textheight\else\Gin@nat@height\fi}
\makeatother
% Scale images if necessary, so that they will not overflow the page
% margins by default, and it is still possible to overwrite the defaults
% using explicit options in \includegraphics[width, height, ...]{}
\setkeys{Gin}{width=\maxwidth,height=\maxheight,keepaspectratio}
\IfFileExists{parskip.sty}{%
\usepackage{parskip}
}{% else
\setlength{\parindent}{0pt}
\setlength{\parskip}{6pt plus 2pt minus 1pt}
}
\setlength{\emergencystretch}{3em}  % prevent overfull lines
\providecommand{\tightlist}{%
  \setlength{\itemsep}{0pt}\setlength{\parskip}{0pt}}
\setcounter{secnumdepth}{5}
% Redefines (sub)paragraphs to behave more like sections
\ifx\paragraph\undefined\else
\let\oldparagraph\paragraph
\renewcommand{\paragraph}[1]{\oldparagraph{#1}\mbox{}}
\fi
\ifx\subparagraph\undefined\else
\let\oldsubparagraph\subparagraph
\renewcommand{\subparagraph}[1]{\oldsubparagraph{#1}\mbox{}}
\fi

%%% Use protect on footnotes to avoid problems with footnotes in titles
\let\rmarkdownfootnote\footnote%
\def\footnote{\protect\rmarkdownfootnote}

%%% Change title format to be more compact
\usepackage{titling}

% Create subtitle command for use in maketitle
\providecommand{\subtitle}[1]{
  \posttitle{
    \begin{center}\large#1\end{center}
    }
}

\setlength{\droptitle}{-2em}

  \title{Fall 2018 Mid Term Solution}
    \pretitle{\vspace{\droptitle}\centering\huge}
  \posttitle{\par}
    \author{Chen Wang\footnote{Undergraduate in Computer Engineering, Samueli School
  of Engineering, University of California, Irvine.
  (\href{mailto:chenw23@uci.edu}{\nolinkurl{chenw23@uci.edu}})}}
    \preauthor{\centering\large\emph}
  \postauthor{\par}
      \predate{\centering\large\emph}
  \postdate{\par}
    \date{11/8/2019}


\begin{document}
\maketitle

{
\setcounter{tocdepth}{3}
\tableofcontents
}
\hypertarget{os-interfaces}{%
\section{OS Interfaces}\label{os-interfaces}}

\textbf{\emph{This question is not covered in this midterm}}

\hypertarget{basic-page-tables}{%
\section{Basic page tables}\label{basic-page-tables}}

\hypertarget{points-draw-page-table-structure}{%
\subsection{(5 points) Draw page table
structure}\label{points-draw-page-table-structure}}

Alice wants to construct a page table that maps virtual addresses 0x0,
0x1000and 0x2000 into physical addresses 0x1000, 0x2000, and 0x3000.
Assume that the Page Directory Page is at physical address 0x0, and the
Page Table Page is at physical address0x00001000 (which is PPN 0x00001).

Draw a picture of the page table Alice will construct (or alternatively
simply write it down in the format similar to the one below):

Page Directory Page:

\texttt{PDE\ 0:\ PPN=0x1,\ PTE\_P,\ PTE\_U,\ PTE\_W}

\ldots{} all other PDEs are zero

The Page Table Page:

\begin{verbatim}
PTE 0: PPN=0x1, PTE_P, PTE_U, PTE_W
PTE 1: PPN=0x2, PTE_P, PTE_U, PTE_W
PTE 2: PPN=0x3, PTE_P, PTE_U, PTE_W
\end{verbatim}

\ldots{} all other PTEs are zero

\textbf{\emph{Reference Solution:}}

Page Directory Page:

\texttt{PDE\ 0:\ PPN=0x1,\ PTE\_P,\ PTE\_U,\ PTE\_W}

\ldots{} all other PDEs are zero

The Page Table Page:

\begin{verbatim}
PTE 0: PPN=0x1, PTE_P, PTE_U, PTE_W
PTE 1: PPN=0x2, PTE_P, PTE_U, PTE_W
PTE 2: PPN=0x3, PTE_P, PTE_U, PTE_W
\end{verbatim}

\ldots{} all other PTEs are zero

\hypertarget{stack-and-calling-conventions}{%
\section{Stack and calling
conventions}\label{stack-and-calling-conventions}}

Alice developed a program that has a function \texttt{foo()} that is
called from two other functions \texttt{bar()} and \texttt{baz()}:

\begin{verbatim}
int foo(int a) {
  ...
}

int bar(int a, int b) 
  ...
  foo(x);
  printf("bar:%d\n", x);
  ...
}

int baz(int a, int b, int c) {
  ...
  foo(x);
  printf("baz:%d\n", x);
  ...
}
\end{verbatim}

While debugging her program Alice observes the following state when she
hits a breakpoint of the program inside \texttt{foo()} (assume that the
compiler does not inline invocations of \texttt{foo()}, \texttt{bar()},
and \texttt{baz()}, and follows the calling conventions that we've
covered in the class)

\begin{verbatim}
The bottom of the stack:

0x8010b5b4: ...
0x8010b5b0: 0x00000003
0x8010b5ac: 0x00000002
0x8010b5a8  0x80102e80
0x8010b5a4: 0x8010b5b4
0x8010b5a0: 0x80112780
0x8010b59c: 0x00000001
0x8010b598: 0x80102e32
0x8010b594: 0x8010b5a4    <-- ebp
0x8010b590: 0x00000000    <-- esp
\end{verbatim}

\hypertarget{explain-stack}{%
\subsection{Explain Stack}\label{explain-stack}}

\begin{enumerate}
\def\labelenumi{(\alph{enumi})}
\tightlist
\item
  (5 points) Provide a short explanation for each line of the stack dump
  above (you canannotate the printout above).
\end{enumerate}

\textbf{\emph{Reference Solution}}

\begin{verbatim}
The bottom of the stack:

0x8010b5b4: ...        // ebp
0x8010b5b0: 0x00000003 // argument #2 to the function that called foo()’s caller
0x8010b5ac: 0x00000002 // argument #1 to the function that called foo()’s caller
0x8010b5a8  0x80102e80 // return address
0x8010b5a4: 0x8010b5b4 // ebp
0x8010b5a0: 0x80112780 // (local variable, argument to a funciton, or register
                          spill inside function that called foo)
0x8010b59c: 0x00000001 // arg to foo
0x8010b598: 0x80102e32 // return address for foo()
0x8010b594: 0x8010b5a4    <-- ebp
0x8010b590: 0x00000000    <-- esp (local variable, argument to a funciton, or 
                                    register spill inside foo)
\end{verbatim}

\hypertarget{analyze-output}{%
\subsection{Analyze output}\label{analyze-output}}

\begin{enumerate}
\def\labelenumi{(\alph{enumi})}
\setcounter{enumi}{1}
\tightlist
\item
  (5 points) If Alice continues execution of her program what output
  will she see on thescreen (justify your answer).
\end{enumerate}

\textbf{\emph{Reference Solution}}

We know that foo() can be called from bar() or baz(), but we also know
that the caller of foo()'s caller i.e., either bar() or baz(), got two
arguments. Hence, it's bar(). And since we know that foo() got 0x1 as
argument the string Alice will see on the screen should be
\texttt{bar:1}

\hypertarget{xv6-process-organization.}{%
\section{Xv6 process organization.}\label{xv6-process-organization.}}

In xv6, in the address space of the process, what does the following
virtual addresses contain?

\hypertarget{points-virtual-address-0x0}{%
\subsection{(3 points) Virtual address
0x0}\label{points-virtual-address-0x0}}

\textbf{\emph{Reference Solution}}

The memory at virtual address 0x0 contains the text section (code) of
the user process.

\hypertarget{points-virtual-address-0x80100000}{%
\subsection{(3 points) Virtual address
0x80100000}\label{points-virtual-address-0x80100000}}

\textbf{\emph{Reference Solution:}}

The memory at virtual address 0x80100000 contains the text section
(code) of the kernel. During the boot the kernel was loaded at physical
address 0x100000 (1MB) and then later this address was mapped at 2GBs +
1MB or (0x80000000 + 0x100000).

\hypertarget{points-what-physical-address-is-mapped-at-virtual-address-0x80000000}{%
\subsection{(3 points) What physical address is mapped at virtual
address
0x80000000}\label{points-what-physical-address-is-mapped-at-virtual-address-0x80000000}}

\textbf{\emph{Reference Solution:}}

Physical address 0x0.

\hypertarget{physical-address-mapping-lookup}{%
\subsection{Physical address mapping
lookup}\label{physical-address-mapping-lookup}}

(7 points) Is there a way for the kernel to find out what physical
address is mapped at a specific virtual address? Provide an explanation
and a code sketch (pseudocode is ok, no need to worry about correct C
syntax). Your code should take a virtual address as an input and resolve
it into the physical address that is mapped into that virtual address by
the process page table (in your code feel free to re-use functions that
are already implemented in the xv6 kernel).

\textbf{\emph{Reference Solution:}}

In xv6 we can access the entire page table and the page tables contain
information about how a virtual address mapps to the physical address.
Therefore, we only need to go though the table to find out where the
physical page lie in the page table and then we will be able to find out
the virtual addresses that are directing to this physical address.

Please see the C file at
\url{https://github.com/StrayBird-ATSH/OperatingSystemCourseMidTermExams/blob/master/Fall\%202018/Fall2018Mid-Virtual2Physical.c}

\hypertarget{protection-and-isolation}{%
\section{Protection and isolation}\label{protection-and-isolation}}

\hypertarget{kenrel-memory-protection-explanation}{%
\subsection{Kenrel Memory Protection
Explanation}\label{kenrel-memory-protection-explanation}}

(5 points) In xv6 all segments are configured to have the base of 0 and
limit of 4GBs, which means that segmentation does not prevent user
programs from accessing kernelmemory. Nevertheless, user programs can't
read and write kernel memory. How (through what mechanisms) such
isolation is achieved?

\textbf{\emph{Reference Solution:}}

Above all, xv6 adopts page tables. That is, the kernel memory and user
memory will reside in different pages. Also, each page has a flag
indicating whether this page is for kernel or for user. Therefore, when
a user program wants to access kernel program, it will access the kernel
pages, and visiting a page with a kernel flag will trigger a fault. So
achieved.

\hypertarget{system-calls}{%
\section{System calls}\label{system-calls}}

\textbf{\emph{This question is not covered in this midterm}}


\end{document}
